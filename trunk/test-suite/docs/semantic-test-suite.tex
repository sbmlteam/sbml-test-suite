\documentclass{cekarticle}
\usepackage{color}
\usepackage{amsmath}
\usepackage{amssymb}
\usepackage{array}

\begin{document}

%=============================================================================
% Title page
%=============================================================================

\title{SBML Semantic Test Suite\\User Guide and Reference}

\author{Andrew Finney}

\authoremail{afinney@caltech.edu}

\date{Last update: 29 June 2007}

\maketitlepage

\section{Introduction}
\label{sec:introduction}

This document describes how to use the SBML semantic test suite.

\subsection{Aims and Scope of the Semantic Test Suite}

The semantic test suite provides a set of valid SBML models each
with a simulated time course data.  This test suite is designed to
be used by developers to check that their simulators produce
results that are consistent with the SBML standard and thus with
each other.  Thus the overall aim of the semantic test suite is to
facilitate the consistent interpretation of SBML models.

\subsection{Semantic Test suite Documentation and Download}

This document does not provide details of specific models in the
test suite. This is provided by the suite's pages on the SBML
wiki, see \url{http://sbml.org/wiki/}. However
Section~\ref{sec:structure} describes in general terms how the
suite is structured.  The suite and scripts to automate test
application are available from the SBML project CVS repository
on SourceForge (\url{http://sourceforge.net/projects/sbml}.

\section{Generic Structure of Semantic Test Suite}
\label{sec:structure}

The test suite is divided into categories with a top level
directory for each category.  In addition there are two other top
level directories: \texttt{docs} and \texttt{bin}
containing documentation and automation scripts, respectively. Each
category directory contains a subdirectory for each individual
test.  The only exception to this is the \texttt{basic-reactions}
category which is divided into models each of which has a set of
individual variant test directories.

\subsection{Content of Each Test Directory}
\label{sec:test-directory}

Each test directory contains the following files:

\begin{itemize}

\item \texttt{<model>-l2.xml} the SBML Level 2 test model

\item \texttt{<model>.CSV} comma separated times series simulation
results.  The first line of this file doesn't contain data but
identifies the columns of data.  The first column is time.  The
subsequent columns are all species concentrations.  Described in
more detail in Section~\ref{sec:formats}.

\item \texttt{<model>.html} documentation on the model.

\item \texttt{<model>.GIF} linear scale plot of species
concentrations against linear time scale.

\item \texttt{<model>-log.GIF} log scale plot of species
concentrations against linear time scale.

\item \texttt{<model>-diagram.GIF} diagram of model reaction
network.

\item \texttt{<model>.m} Mathematica script used to generate the
above files from the test model.  This script assumes that
Mathematica has automatically loaded the MathSBML package.  (See
\url{http://sbml.org/software/mathsbml}).

\item \texttt{<model>.test} file for driving test automation.
Described in more detail in Section~\ref{sec:formats}.

\end{itemize}

The \texttt{<model>} file name is designed to be unique across the
suite and is composed from the category and test names.

\section{Automation}
\label{sec:automation}

\subsection{User Guide}
\label{sec:user-guide}

\subsubsection{Platform and Environment requirements}

The suite automation scripts run in a UNIX style environment. On
Windows this means that you need to install Cygwin and then run
commands in the Cygwin bash shell.

\subsubsection{Writing a Simulator Wrapper}

The test suite comes with a set of scripts to enable the testing
of a given simulator.  Before you can start running automated
tests you must write a script or program that `wraps' your
simulator.  The wrapper script should be written so that one of
the script runs one simulation by the wrapped simulator.  Such a
wrapper should take the following arguments in order:
\begin{itemize}

\item the path of the SBML file containing the model

\item the simulation time in seconds

\item the number of simulation time steps.  For each time step a
row of simulation output should be generated.  The time interval
between simulation steps is the simulation time divided by the
number of steps.

\item the filename to use when storing the simulation output.

\item the temporary directory in which to insert any temporary
files

\item the remaining arguments are the names of the species whose
concentration values should be placed in the simulation output.
\end{itemize}

The simulation output must be in ASCII comma separated value (CSV)
format (described in more detail in Section~\ref{sec:formats}. The
first row can contain any header information you like and will be
ignored. The second row should contain initial condition data for
$t=0$. There should then be one row output for each time step. For
example if the test requires 50 time steps then the simulation
output should consist of a CSV file containing 52 rows.  The first
column of the file should be the simulation time for the timestep.
The remaining columns should be species concentrations at that
time in the order given in the arguments to the wrapper. The
\texttt{*.CSV} files in the test directories are examples of the
format required.  An example wrapper script for MathSBML is
\texttt{bin/wrappers/mathSBML-wrapper.bsh}.

To test your simulator wrapper, first make sure the Test Suite's \texttt{bin}
directory is on the PATH and locate a specific \texttt{*.test}
file.  You can then run that test on the simulator with the
command \texttt{test.bsh <wrapper> <*.test file>}.  No output
means that the command was successful.

\subsubsection{Running Through the Complete Test Suite}
Once you have a wrapper you can test your simulator by running the
tests in the suite.  The simplest way to do this is to change
directory to the top of the test suite and type
\texttt{./runtests.bsh -vwrapper="<wrapper>"}. If you add the
argument \texttt{-vhaltOnFailure="true"} then the test sequence
will stop when the first failure is encountered.

\subsubsection{Error Messages from Failed Tests}
When running test scripts one of three error messages will be
generated:

\begin{itemize}

\item \texttt{<model>.CSV and testout.csv have different numbers
of lines} This indicates that the wrapper is not interpreting the
number of time steps correctly and not generating the correct
number of rows in the simulation output.

\item \texttt{time fields out of sync on line <line number> in
file <model>} This indicates that the time column of the
simulation output is incorrect because for example the time is
being incorrectly calculated or the file is incorrectly formatted.

\item \texttt{species <n> values <x> and <y> don't match at <time>
on line <line number> in file\\ <filename>} This message indicates
that one of the species concentration values generated by the
wrapper isn't close enough to the original results generated from
the model. This normally indicates that either

\begin{itemize}

\item the wrapped simulator is not interpreting the SBML contained
in the model correctly;

\item the wrapped simulator is not as accurate as it could be; or

\item the original simulator is not as accurate as it could be.

\end{itemize}
\texttt{n} is the number of the first species that doesn't match
on the given line.  The species are indexed in the order given in
the corresponding test file.  The file given in the message is a
\texttt{*.CSV.join} file whose format is decibed in
Section~\ref{sec:formats}.
\end{itemize}

When any of the above error messages are generated they are
followed by a very short description of the test and the URL of
detailed documentation on the test.

\subsubsection{Running your own test sets}

It is possible to run subsets of the test suite in a custom order
with a custom categorization.  Simply make a copy of the
\texttt{bin/testlist.txt} file and edit the copy to create
your test list file. The format of this file should be self
evident however see Section~\ref{sec:formats} for the details.
Your test list file must have at least one category and there
should at least one test in each category.  A category declaration
should precede the tests in that category.  Tests can be placed in
any order. Categories don't have to correspond to the test suite
directory structure.

Once you have edited your test list file you can run the test
sequence declared in that file by placing the \texttt{bin}
directory on the \texttt{PATH} and invoking the command
\texttt{runthroutests.awk -vwrapper="<wrapper>" <test list file>}.
If you add the argument \texttt{-vhaltOnFailure="true"} then the
test sequence will stop when the first failure is encountered.

\subsection{Reference} \label{sec:auto-reference}

This section describes in detail the scripts that comprise the
test automation system and the file formats that they use.

\subsubsection{Scripts}

All the following scripts are located in the \texttt{bin} directory
of the Test Suite.

\textbf{compare.awk} Argument: $<$.join file$>$

Runs through a join file (see Section~\ref{sec:formats} for more
details on the file format), produced by the
\texttt{cvs-comparator.bsh} script, and compares the columns.
Outputs errors if there is significant numerical differences
between the 2 sets of data.

\textbf{createargs.awk} Argument: $<$.test file$>$

Extracts simulation configuration data from the given test file
and returns a string for use as arguments to the simulation
wrapper. Outputs a string of the form: \texttt{$<$time$>$
$<$steps$>$ testout.csv temp $<$species1$>$ $<$species2$>$ ...
$<$speciesN$>$}

\textbf{csv-compartor.bsh} Arguments: $<$CSV File A$>$ $<$CSV File
B$>$

Compares the content of the 2 given files.  Outputs errors if
either file contains different numbers of line or if there is
significant numerical differences between the 2 sets of data.

\textbf{findmissingtest.bsh} Argument: $<$test list file$>$

Compares the content of the test list file with the test files
that exist in the test suite.  Must be executed from the top level
test suite directory.  Outputs those files that are in the test
suite but are not in the given test list.  This output is in a
form that can be pasted into another test list file.

\textbf{generateMessage.awk} Argument: $<$.test file$>$

Extracts model information from the given test file and outputs an
message based on it.

\textbf{number.awk} Argument: $<$.CSV file$>$

Outputs a space separated version of the given file and inserts an
additional first column which is the number of the record.

\textbf{runthroutests.awk} Arguments:
\texttt{-vwrapper="}$<$simulator wrapper$>$\texttt{"} \newline
[\texttt{-vhaltOnFailure="}$<$boolean value$>$\texttt{"}] $<$test
list file$>$

Runs the test sequence given in the given test list file using the
given simulator wrapper.  If \texttt{haltOnFailure} is set to
`true' then the sequence stops on the first test that fails.  Each
test should have an associated \texttt{*.CSV} and
\texttt{*-l2.xml} file both with the same name and located in the
same directory. This script must be run from the top level
directory of the test suite.

\textbf{test.bsh} Arguments: $<$simulator wapper$>$ $<$.test
file$>$

Runs the given test using the given simulator wrapper.  This
script expects a \texttt{*.CSV} file and a \texttt{*-l2.xml} file
both in the same directory and having the same name as the test
file.  The wrapper is passed the path to the \texttt{*-l2.xml}
file together with simulation control arguments from the
\texttt{.test} file.  The results of the simulation are compared
with the content of the \texttt{*.CSV} file.

The following script is located in the top level directory of the
test suite

\textbf{runtests.bsh} Arguments: \texttt{-vwrapper="}$<$simulator
wrapper$>$\texttt{"} [\texttt{-vhaltOnFailure="}$<$boolean
value$>$\texttt{"}]

Runs the complete test suite against the given simulator wrapper.
This script must be run from the top level directory of the test
suite.

\subsubsection{File Formats}
\label{sec:formats}

\textbf{*.csv or *.CSV}

These files contain time series simulation output data in comma
separated value format (which is readable for example by Microsoft
Excel).  The first row should ignored (it usually contains column
headings for subsequent rows).  The remaining rows comprise the
times series data starting with a row for $t=0$.  The first column
contains the simulation time for each row.  Each remaining column
contains the concentration value for a given species.  The order
of these columns follows that given in the corresponding
\texttt{*.test} file.

\textbf{*.join}

These files contain the `join' (simple combination) of two
\texttt{*.num} files. The join is by the first column of the
\texttt{*.num} files.  The result is a space separated values file
in which the columns have the pattern $<$row number$>$ $<$time$>$
$<$species 1$>$...$<$species N$>$ $<$row number$>$ $<$time$>$
$<$species 1$>$...$<$species N$>$. The first row can be ignored
and just contains column headings.

\textbf{*.num}

These files are simply a space separated values version of a comma
separated values file with an additional first column which is the
row number.

\textbf{*.test}

A \texttt{*.test} file represents a single test of a SBML
simulator. A \texttt{<name>.test} file always has an associated
model file and `correct' simulation results file both in the same
directory with respective corresponding names
\texttt{<name>-l2.xml} and \texttt{<name>.CSV}.  The
\texttt{*.test} file itself contains information on how to execute
the simulation test together with information on the purpose of
test and/or the key features of the associated model.

A \texttt{*.test} file consists of a series of records where each
record is on a separate line.  Each record begins with a keyword
which indicates the type of the record.  The rest of the record is
separated from the keyword by at least one space.  The keywords
and the corresponding record content are:

\begin{itemize}

\item TIME $<$simulation time in seconds$>$

\item STEPS $<$number of simulation steps$>$ - excludes the first
data set at $t=0$

\item SPECIES $<$space separated sequence of species
identifiers$>$ - corresponds to the columns of the associated CSV
file and indicates the required column order of the CSV file
generated by a simulator wrapper

\item URL $<$web page name (last part of URL) which describes
test$>$ - Currently this contains the SBML wiki page name for the
page containing information on this specific test.  (A complete
URL is not used so that information on the test suite can be
easily relocated.)

\item REM $<$text$>$ - short summary text describing the purpose
of the test and or specific content of the associated model.

\end{itemize}

A \texttt{*.test} file must contain at least one record of each
type.  A \texttt{*.test} file can contain any number of REM
records but only one record each of the other types. The order of
REM records is significant: the text in these fields is combined
to form a complete text message. The order of the other record
types is not significant.

\textbf{test list}

A test list file defines of a sequence of simulation tests which
are divided into categories. A \texttt{*.test} file consists of a
series of records where each record is on a separate line. Each
record begins with a keyword which indicates the type of the
record.  The rest of the record is separated from the keyword by
at least one space.  The keywords and the corresponding record
content are:

\begin{itemize}

\item TEST $<$test file$>$ - Path of a \texttt{*.test} file
relative to the top level directory of the test suite.

\item CATEGORY $<$category name$>$

\end{itemize}

A category record should precede the test records for tests placed
in that category. A test list file must have at least one category
record and there should at least one test record immediately
following  each category record. Test records can be placed in any
order. Categories don't have to correspond to the test suite
directory structure.

%\newpage
%\section{Appendix}
%\setcounter{secnumdepth}{2}
%\appendix
%
%\section{Elements introduced in this proposal}
%\section{Attributes introduced in this proposal}
%=============================================================================
% References
%=============================================================================
%\newpage
%\bibliographystyle{apalike}
%\bibliography{strings,a,b,c,d,e,f,g,h,i,j,k,l,m,n,o,p,q,r,s,t,u,v,w,x,y,z}
\end{document}
